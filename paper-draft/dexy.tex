\documentclass{article}   % list options between brackets

\usepackage{color}
\usepackage{graphicx}
%% The amssymb package provides various useful mathematical symbols
\usepackage{amssymb}
%% The amsthm package provides extended theorem environments
%\usepackage{amsthm}
\usepackage{amsmath}

\usepackage{listings}

\usepackage{hyperref}

\usepackage{systeme}

\def\shownotes{1}
\def\notesinmargins{0}

\ifnum\shownotes=1
\ifnum\notesinmargins=1
\newcommand{\authnote}[2]{\marginpar{\parbox{\marginparwidth}{\tiny %
  \textsf{#1 {\textcolor{blue}{notes: #2}}}}}%
  \textcolor{blue}{\textbf{\dag}}}
\else
\newcommand{\authnote}[2]{
  \textsf{#1 \textcolor{blue}{: #2}}}
\fi
\else
\newcommand{\authnote}[2]{}
\fi

\newcommand{\knote}[1]{{\authnote{\textcolor{green}{Alex notes}}{#1}}}
\newcommand{\mnote}[1]{{\authnote{\textcolor{red}{Amitabh notes}}{#1}}}

\usepackage[dvipsnames]{xcolor}
\usepackage[colorinlistoftodos,prependcaption,textsize=tiny]{todonotes}


% type user-defined commands here
\usepackage[T1]{fontenc}

\usepackage{flushend}

\newcommand{\bc}{ERG}
\newcommand{\stc}{stablecoin}
\newcommand{\sct}{stablecoin}
\newcommand{\dx}{Dexy}
\newcommand{\dxg}{DexyGold}

\newcommand{\ma}{\mathcal{A}}
\newcommand{\mb}{\mathcal{B}}
\newcommand{\he}{\hat{e}}
\newcommand{\sr}{\stackrel}
\newcommand{\ra}{\rightarrow}
\newcommand{\la}{\leftarrow}
\newcommand{\state}{state}

\newcommand{\ignore}[1]{} 
\newcommand{\full}[1]{}
\newcommand{\notfull}[1]{#1}
\newcommand{\rand}{\stackrel{R}{\leftarrow}}
\newcommand{\mypar}[1]{\smallskip\noindent\textbf{#1.}}

\begin{document}

\title{Dexy - Simple Stablecoin Design Based On Algorithmic Central Bank}
\author{kushti, scalahub}


\maketitle

\begin{abstract}
In this paper, we consider a new stablecoin protocol design called \dx. The protocol is maintaining the peg with two
assets only~(base currency and stablecoin), and also two functional components, namely, a reference market~(in form of a liquidity pool
where anyone can trade base currency against stablecoin), and an algorithmic central bank, which is responsible for new
stablecoins issuance, and also for maintaining stablecoin pricing corresponding to a peg in the reference market.
\end{abstract}


\section{Introduction}

Algorithmic stablecoins is a natural extension of cryptocurrencies, trying to 
solve problems with volatility of their prices by pegging stablecoin price to an
asset which price is considered to be more or less stable over long periods of time (e.g. gold).

Having an asset with a stable value could be useful in many scenarios, for example:
\begin{itemize}
\item securing fundraising; a project can be sure that funds collected during fundraising will have stable value in the mid- and long-term.
\item doing business with predictable results. For example, a shop can be sure that funds collected from sales will be about the same value when the shop is ordering goods from warehouses~(otherwise, the shop may go bankrupt if its margin is not that big to cover exchange rate fluctuations). 
\item shorting: when cryptocurrency prices are high, it is desirable for investors to rebalance their portfolio by increasing exposure to fiat currencies (or real-world commodities). However, as fiat currencies and centralized exchanges impose significant risks, it would be better to buy fiat and commodity substitues in form of stablecoins on decentralized exchanges.
\item lending and other decentralized finance applications. Stability of collateral value is critical for many applications.
\end{itemize}

Algorithmic stablecoins are different from centralized stablecoins, such as USDT and USDC, for which there is a trusted party which holds the peg. In case of an algorithmic stablecoin, the 
pegging is done via rebasement of total supply~(as in Ampleforth etc), or via imitating a trusted party which holds reserves and doing market interventions when it is needed for getting exchange rate back to the peg. Imititating the trusted party is usually done by allowing anyone on the blockchain to create over-collateralized financial instruments, such as collateralized debt positions~(as 
in DAI), zero-coupon bonds~(as in the Yield protocol), reserve asset~(as in Djed), or by issuing stabilizing financial instruments in case of depeg~(as in Neutrino). \knote{add links to the paragraph above}

In this work we present \dx{}, a stablecoin protocol where an algorithmic central bank performing interventions in case of depeg is presented explicitly as a contract with few predefined rules. The bank is trying to stabilize stablecoin value on the markets, using a liquidity pool as a reference market, by providing stablecoin liquidity, when stablecoin price is over the peg, or injecting base currency from its reserves, when the stablecoin is under the peg. In extreme case, when bank reserves are depleted and stablecoin is still under the peg, its value is restored by extracting stablecoins from the liquidity pool~(to inject back when price will go above the peg again). In some aspects \dx{} could resemble Fei or Gyroscope. \knote{make comparison subsection}  

The rest of the paper is organized as follows.  In Section~\ref{design-general} we sketch \dx{} design in general and
introduce notation used in other sections. Section~\ref{worst-case} is defining worst-case scenario for bank reserves~(
and so, for stablecoin stability). Then \dx{} is detalized in Section~\ref{detailed-rules}. Stability of the rules are
discussed in Section~\ref{stability}. Trust assumptions \dx{} protocol is based on are stated in Section~\ref{kya}.
Section~\ref{dexygold} describes concrete \dx{} implementation, gold-pegged DexyGold stablecoin and its
tokenomics.

\section{\dx{} Design in General}
\label{design-general}

Unlike popular algorithmic stablecoins based on two tokens~(in addition to base currency), such as Djed, \dx{} is based on one token but two protocols. In the first place,
there is a reference market~(done as on-chain protocol, such as automated market maker~\cite{xu2021sok}), where trading of \dx{} vs the base currency (\bc{}) happens. In the second place, to tackle with situations when the reference market price is way too different from a target price (price of a pegged asset, as reported by a trusted oracle), there is an algorithmic central bank which is doing interventions in order to readjust the market price~(so make it closer to the oracle's one). The central bank can also mint new \dx{} tokens by selling them for \bc{}. The bank is using reserves in \bc{} it is buying for interventions. 

As a simple solution for the {\em reference market}, we are using constant-product Automated Market Maker (CP-AMM) liquidity pool, similar to Spectrum and UniSwap. The pool has \bc{} on one side and \dx{} on another. For CP-AMM pool, multiplication of \bc{} and \dx{} amounts before and after a trade should be preserved, so $e * u = e' * u'$, where $e$ and $u$ are amounts of \bc{} and \dx{} in the pool before the trade, and $e'$ and $u'$ are amounts of \bc{} and \dx{} in the pool after the trade, correspondingly. As for any CP-AMM pool, it is possible to put liquidity into the pool, and remove liquidity from it, however, there are some limitations here we are going to uncover further. 

The bank has two basic operations. It can mint new \sct{} tokens per request, using trusted oracle's price, by accepting \bc{} in its reserves. It also can intervene into markets by providing \bc{} from reserves when needed~(namely, when price in the pool $\frac{u}{e}$ is significantly different from price on external markets $p$ which reported by oracle).

Now we are going to consider what are possible failure scenarios for such design and how to put restrictions and design rules for the system to ensure stable pricing for \sct{} tokens. 

\subsection{Notation}

We are introducing notation used further:
\begin{itemize}
  \item{} $T$ - period before intervention starts. After one intervention the bank can start another one only after $T$ time units passed. 
  \item{} $p$ - price reported by the oracle at the moment (for example, 20 USD per ERG)
  \item{} $s$ - price which the bank should stand in case of sudden price crash. For example, we can assume that $s = \frac{p}{4}$ (so if p is 20 USD per ERG, then $s$ is 5 USD per ERG, means the bank needs to have enough reserves to save the markets when the price is suddenly crashing from 20 to 5 USD per ERG)
  \item{} $R$ - ratio between $p$ and $s$, $R = \frac{p}{s}$
  \item{} $r$ - ratio between $p$, and price in the pool, which is $\frac{u}{e}$, thus $r = \frac{p}{\frac{u}{e}} = \frac{p*e}{u}$
  \item{} $e$ - amount of \bc{} in the liquidity pool 
  \item{} $u$ - amount of \stc{} in the liquidity pool
  \item{} $O$ - amount of \stc{} outside the liquidity pool. The distribution in $O$ is not known for the \dx{} protocol, but the bank can easily store how many \sct{} tokens were minted and then get $O$ by deducting $u$ from it.
  \item{} $E$ - amount of \bc{} in the bank. 
\end{itemize}  

\section{Worst Case Scenario and Bank Reserves}
\label{worst-case}

The bank is doing interventions when the situation is far from normal in the markets, and enough time passed for markets to stabilize themselves with no interventions. In our case, the bank is doing interventions based on stablecoin price in the liquidity pool in comparison with oracle provided price. The bank's intervention then is about injecting its \bc{} reserves into the pool.  

First of all, let's assume that the oracle price crashed from $p$ to $s$ sharply and stands there, and before the crash there were $e$ of \bc{} and $u$ of \sct{} in the liquidity pool, with pool's price being $p$. The worst case then is when no liqudity put into the pool during the period $T$. With large enough $T$ and large enough $R$ this assumption is not very realistic probably: some traders will buy \bc{}s with their \sct{}s anyway, price is usually failing with swings where traders could mint \sct{}s thus increasing \bc{} bank reserves, etc. However, it would be reasonable to consider worst-case scenario, then in the real world \dx{} will be even more durable than in theory. 

In this case, the bank must intervene after $T$ units of time, as the price differs significantly, and restore the price in the pool, so set it to $s$. We denote amounts of \bc{} and \sct{} in the pool after the intervention as $e'$ and $u'$, respectively. Then:

\begin{itemize}
  \item{} $e * u = e' * u'$
  \item{} as the bank injects $E_1$ ergs into the pool, $e' = e + E_1$
  \item{} $\frac{u'}{e'} = s$, thus $u' = s * (e + E_1)$ 
  \item{} from above, $E_1 = \sqrt{\frac{e * u}{s}} - e$
\end{itemize}

So by injecting $E_1$ \bc{}, the bank recovers the price. However, this is not enough, as now there are $O$ \sct{} units which can be injected into the pool from outside. 
Again, in the real life it is not realistic to assume that all the $O$ \sct{} would be injected, as some of them are simply lost, some would be kept to buy cheap ERG at the bottom~(we remind that \sct{} often used as a bet for \bc{} price decline), etc. However, we need to assume worst-case scenario again. We also unrealistically assume that all the $O$ tokens are being sold in very small batches not significantly affecting price in the pool, and after each batch seller of a new batch is waiting for a bank intervention to happen (so for $T$ units of time), and sells only after the intervention. In this case all the $O$ tokens are being sold at price close to $s$, so the bank should have about $E_2 = \frac{O}{s}$ \bc{}s in reserve to buy the tokens back.

Summing up $E_1$ and $E_2$, we got \bc{} reserves the bank should have to be ready for worst-case scenario: $E_w = E_1 + E_2 = \sqrt{\frac{e * u}{s}} - e + \frac{O}{s}$.

It is simple to see why this scenario is worst-case for the bank. In this scenario, the bank~(and only the bank) is buying all the $O$ of external $\sct{}$ at the worst possible price $s$, and also set the price by burning its own reserves only.  

\section{Bank and Pool Rules}
\label{detailed-rules}

Based on needed reserves for worst-case scenario estimation, we can consider minting rules accordingly. Similarly to SigUSD~(a Djed instantiation), we can, for example, target for security in case of 
4x price drop, so to consider $R = \frac{p}{s} = 4$, and allow to mint \sct{}~while there are enough, so not less than $E_w$, \bc{}~in reserves. However, in this case most of time \sct{} would be non-mintable, and only during moments of \bc{} price going up significantly it will be possible to mint \dx{}. As worst-case scenario is based on unrealistic assumptions, unlikely a realistic protocol can be built on top of it.  

Thus we leave worst-case scenario for UI, so dapps working with the \dx{} may show e.g. collateralization for the worst-case scenario. Having on-chain data analysis, more precise estimations of reserve quality can be made~(by considering \sct{}s locked in DeFi protocols, likely forgotten, etc).

Instead, we always allow for cautios minting. That is, we whether allow minting when oracle price is above pool's price~(in this way the bank is providing liquidity for arbitrage when \sct{} pice is above the peg), or we allow to mint a little bit (per some time period) when liquidity pool is in good shape. In details, we have two following minting operations, with minting price being the oracle's price $p$:  

\begin{itemize}
  \item{Arbitrage mint: } if price reported by the oracle is higher than in the pool, i.e. $p > \frac{u}{e}$, we allow to print enough \sct{} tokens to bring the price to $p$. That is, the bank allows to mint up to $\delta_u = \sqrt{p*e*u}-u$ \sct{} by accepting up to $\delta_e = \frac{\delta_u}{p}$ \bc{}s into reserves. 

  \em{To instantiate the rule, we can allow for arbitrage minting if the price $p$ is more than $\frac{u}{e}$ by at least $1\%$ for time period $T_{arb}$ (e.g. 1 hour), also, the bank is charging $0.5\%$ fee for the operation. After arbitrage mint it is not possible to do another one within 30 minutes (to prevent aggressive liquidity minting via chained transactions etc).}

  \item{Free mint: } we allow to mint up to $\frac{u}{100}$ \sct{}s within time period $T_{free}$. 
  \em{To instantiate the rule, we propose to allow for free mint if $0.98 < r < 1.02$. Minting fee in this case is also $0.5\%$. We propose to set $T_free$ to $12$ hours, then bank reserves can grow by $2\%$ of LP volume per day when the peg is okay.}.
\end{itemize}  

 

We also state following rules for the liquidity pool (which, otherwise, acts as ordinary CP-AMM liquidity pool): 

\begin{itemize}
   \item{Stopping withdrawals: } if $r$ is below some threshold, withdrawals are stopped, so only swaps are possible.  
   \em{To instantiate the rule, we propose to stop withdrawals immediately if $r <= 0.98$.}

   \item{Second stabilization mechanism: } what if the bank is out of reserves, but \sct{} is still below the peg? In this case we restore price in the pool by removing liquidity, and there are two possible options here:

   \begin{enumerate}
   \item{Burn: } if the bank if empty, and $r$ is below some threshold, it is allowed to burn \sct{}s in the pool. 
   \em{To instantiate the rule, we propose to burn \sct{} if $r <= 0.95$ for time period $T_{burn}$. $T_{burn}$ must be quite big, e.g. $1 {\ week}$. We burn enough to return to the state of stopped withdrawals, so to $r = 0.98$. After burning the tracker is reset, so another burn will be done sfter $T_{burn}$ at least.}

   \item{Extract for the future: } if the bank is empty and $r$ is below some threshold, it is allowed to extract \sct{}s from the pool and lock by a contract which is releasing \sct{}s in the future when \sct{} price is above the peg.
   \em{To instantiate the rule, we propose to extract \sct{} if $r <= 0.95$ for time period $T_{burn}$ (so the same as in burn). 
   To prioritize extracted funds over arbitrage mint, we do not have delay in releasing contract. We burn enough to return to the state of stopped withdrawals, so close to $r = 0.98$~(exactly, $0.97 <= r <= 0.98$). After extraction the tracker is reset, so another burn will be done sfter $T_{burn}$ at least.}
   \end{enumerate}
\end{itemize} 

In addition, we introduce $2\%$ redemption fee to avoid excessive liquidity hopping.

\section{Stability}
\label{stability}

With second stabilization mechanism~(burning or extraction) in place, the price in the reference market will be eventually stabilized. However, for liquidity holders burning is painful, extraction not so but still not desirable, thus \dx{} protocol is trying to avoid it~(unlike other protocols, such as Gyroscope or Fei, where redemption rate fails below $1$ immediately as collateralization falls under $100\%$), by giving markets time to self-stabilize, and then doing interventions. This could mean slower stabilization, in comparison with other protocols. Slower stabilization helps the bank to play with possible adversaries in the environment with information assymetry~(humans always have access to more information than algorithmic bank, with more flexible decision making as well).

\dx{} is also cautios about minting new \sct{}s, which could mean slow growth of number of tokens issued. This could be inconvient, especially for big players, but the protocol is focused on stability in the first place. 

Please note, that liquidity pool is disincentivizing massive bank runs due to its constant-product nature. Actually, for the bank massive bank runs are simpler for bank to resolve, in comparison with slow drain as in worst-case scenario. 


\section{Trust Assumptions}
\label{kya}

It is important to explicitly state assumptions the protocol is based on, so a user can choose whether to trust it or not.

\begin{itemize}
  \item{Oracle: } the biggest trust issue in the protocol is oracle delivering gold price. This trust issue is unavoidable but can be relaxed by using not a single entity but a federation of oracles with an average price (after noise filtering) to be delivered to the Dexy central bank.
  It is important to reward oracles from protocol activities~(such as minting), otherwise, they will be incentivized to attack the protocol. See Section~\ref{oracle-incentivization} for details.

  \item{No governance: } unlike most of stablecoin protocols, we do not consider governance, as usually it is another trust issue. Real world instantiations can consider governance, we suppose that such instantiations should clearly inform about governance-related trust assumptions.
\end{itemize}

\section{Dexy Gold}
\label{dexygold}

In this section we consider concrete implementation of the \dx{} protocol, gold-pegged stablecoin named \dxg{}.

\subsection{Oracle incentivization and tokenomics}
\label{oracle-incentivization}

Oracles security is the most important issue for our stablecoin protocol, as this is the only component which is
not working autonomously on-chain. Thus the protocol needs to reward oracles.

Oracle Pools 2.0 framework~\cite{eip23} has support for paying rewards in custom tokens. So we will create GORT (Gold
Oracle Reward Token), which will be used to reward oracles and also developers.

$X$ oracles wil get up to $2 \cdot X$ GORTs per epoch (one datapoint update), per Oracle Pool v2 design~\cite{eip23}~(
$2 \cdot X$ if all the oracles are active). We consider $X = 30$ oracles with one hour long epoch, which means up to $60$
GORTs will be released to oracles.

In addition to oracles, GORT is used to reward developers. To do that, we consider a flat emission contract, which is
releasing the same amount per one hour, so 2 GORT per block during initial period (2 years).

\subsection{Implementation}


\subsection{Bank Contract}
\knote{put contracts here}

\subsection{Simulations}
We made simulations, you can find them in \dx{} repository. \knote{finish}


\section*{Acknowledgments}

Authors would like to thank Ile for his inspiring forum posts.

\bibliography{sources}
\bibliographystyle{ieeetr}

\end{document}
